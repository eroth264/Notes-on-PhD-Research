\section{Notes on Parahoric Groups}

We fix a compact Riemann surface $X$ of genus $g>1$.
The \textit{Narasimhan-Seshadri correspondence} in \cite{ns} found a correspondence between the moduli space of degree $0$, rank $r$, slope-stable holomorphic vector bundles $\mathcal{N}_0$ and irreducible unitary representations of the fundamental group $\pi_1(X)$:

\begin{equation}
\mathcal{N}_0\leftrightarrow\Hom(\pi_1(X),\U(r))/\U(r).
\end{equation}
This correspondence is an isomorphism of complex analytic manifolds.

To generalize this to degree $d$ case of $\mathcal{N}_d$, we need universal central extensions of $\pi_1(X)$. The fundamental group $\pi_1(X)$ has $2g$ generators $A_1,\ldots,A_g,B_1,\ldots,B_g$ with the relation:
\begin{equation}
\prod_{i=1}^g[A_i,B_i]=1
\end{equation}
A universal central extension is a short exact sequence of groups:
\begin{equation}
0\rightarrow\Z\rightarrow\Gamma\rightarrow\pi_1(X)\rightarrow1,
\end{equation}
where $\Z$ maps to a central element $J\in\Gamma$. The group $\Gamma$ has the generators with relations:
\begin{equation}
    \prod_{i=1}^g[A_i,B_i]=J.
    \end{equation}

Using the universal central extension above, any representation $\pi_1(X)\rightarrow\PGL(r,\C)$ lifts to a representation $\Gamma\rightarrow\PGL(r,\C)$. A representation $\pi_1(X)\rightarrow\PGL(r,\C)$ induces a connection with constant central curvature on the  vector bundle $E_\rho$, and we define a subgroup of $\Hom(\Gamma,\GL(r,C))$:
\begin{equation}
\Hom_d(\Gamma,\GL(r,\C))=\{\rho\in\Hom(\Gamma,\GL(r,C))|c_1(E_\rho)=d\}.
\end{equation}

We get a more general \textit{Narasimhan-Seshadri correspondence} for degree $d$:
\begin{equation}
    \mathcal{N}_d\leftrightarrow\Hom_d(\Gamma,\U(r))/\U(r).
\end{equation}

What if $X'\subset X$ had finitely many punctures and we looked at bundles over $X'$? Or more algebraically, what if $X'$ was a quasi-projective smooth algebraic curve over $\C$ (not specifically projective)? Would there still be a Narasimhan-Seshadri correspondence over $X'$?

The introduction of punctures allows for vector bundles on $X'$ to exhibit local behavior around them that was previously impossible only on $X$. One could record this extra information as a flag on the puncture points $X\setminus X'$, with weights, leading to the notion of \textit{parabolic vector bundles}. We follow \cite{mehtasesh} in the first section of this note.

One can then ask how this would generalize to principal-$G$-bundles for a reductive group $G$. It would make sense that principal bundles on $X'$ could have a reduction to a parabolic $H$ on the puncture points $X\setminus X'$. How this works will be explained in the second section, following \cite{pr}.

\subsection{Parabolic Vector Bundles}

As a reminder, a compact Riemann surface $X$ of genus $g>1$ has a universal covering of the upper half plane $H$ (see: uniformization theorem), for which a discrete subgroup $\Gamma$ of $\PSL(2,\R)$ acts (freely, properly discontinuously). $X$ can be recovered as $H/\Gamma$, so that $\Gamma=\pi_1(X)$.

Now instead, we look at $X'=H/\Gamma$ with only finite measure, so $X'$ may have punctures. In \cite{stackex}, parabolic cusps of $\Gamma$ are explained, and we write $H^+=H\cup\textit{Parabolic cusps}$. $X=H^+/\Gamma$ can be viewed as a compact unpunctured Riemann surface containing $X'$.

For a representation $\sigma:\Gamma\rightarrow\U(r,\C)$, we can induce a holomorphic vector bundle $E'_\sigma$ on $X'$ that extends to a holomorphic vector bundle $E_\sigma$ on $X$.

For a parabolic cusp $P\in X\setminus X'$, which is a parabolic cusp in $H^+$, one can find a suitable neighborhood of $P$ appearing like $U/\Gamma_\infty$, where $\Gamma_\infty$ is generated by one element ($z\mapsto z+1$).

In \cite{mehtasesh}, by choosing a basis of $E_\sigma$ at $P$, the representation $\sigma$ can be applied to the generator $z\mapsto z+1$ of $\Gamma_\infty$. The subsequent propositions and corollaries calculate that $E_\sigma$, viewed as an extension of $E'_\sigma$ from $X'$ to $X$ with only $\Gamma$-invariant sections of $E'_\sigma$ extended to $E_\sigma$, comes equipped with a \textit{parabolic structure}.

\begin{definition}
For a compact Riemann surface $X$ with marked points $D_1,\ldots,D_n$ and a vector bundle $E$ on $X$, a \textit{parabolic structure} on $E$ is given by the data on each marked point $D_i$:
\begin{enumerate}[label=(\alph*)]
    \item A flag $0=V_0\subset\ldots\subset V_k=E_{D_i}$
    \item Weights $0\leq\alpha_1<\ldots<\alpha_{k-1}\leq1$
\end{enumerate}
The dimensions of the quotients of the flag $V_{l+1}/V_{l}$ are called the multiplicity $k_l$ of $\alpha_l$.
\end{definition}

One can define morphisms of parabolic vector bundles, which preserve the flag (only when an inequality on the weights are fulfilled). Other constructions include parabolic subbundles, quotients, and more that are similar to that of vector bundles, but respecting the parabolic structure.

Much like for vector bundles, we define a parabolic version of degree. We define such that a parabolic structure $W=E_\sigma$ comes from a unitary bundle if and only if the parabolic degree is $0$.

\begin{definition}

\end{definition}

\subsection{Parahoric Groups}

\subsection{Parahoric $\mathcal{G}$-Torsors}